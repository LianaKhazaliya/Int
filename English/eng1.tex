\documentclass[10pt]{article}

%\pagestyle{plain}
\textwidth=16cm
\textheight=24.0cm
\oddsidemargin=0.3cm
\evensidemargin=0.5cm


\usepackage[utf8]{inputenc}
\usepackage[russian]{babel}
\usepackage{amssymb,amsmath}
\usepackage{fancyhdr}
\usepackage[yyyymmdd,hhmmss]{datetime}
\usepackage{graphicx}
\usepackage[a4paper,bmargin=2.5cm]{geometry}
\usepackage{hyperref}
\voffset=0pt
\headheight=0pt
\headsep=0pt



\begin{document}
\def\chap#1#2{\ \\ {\large\bf#1 \ | \ \tt\scshape#2} \par}

{\bf
\begin{flushright}
\small{Liana Khazaliya\\  \today, \currenttime}
\end{flushright}}
\rm
\hline
\ \\[0.2 cm]
{\large\texttt{Question 1}}
\\
\textit{What was the last really difficult thing you had to do?}
\ \\[0.3cm]
\medskip\par I am not sure that I will tell you about thing which was the most difficult or the last, but I really like it.  
\medskip\par I will start the story from afar. The beginning comes from a summer that was 3 year ago. That summer I participated in tournament in St.Petersburg, where one organizer was carrying printed papers on a unicycle. One month after I was at summer school, where a teacher was going through the territory of a camp with unicycle. And in evenings everyone who wanted could try their abilities. I remember, how after a half of hour I was really killed by attempts to ride a several meters with touching a wall.
\medskip\par  After a half of a year I was in St.Petersburg again and was interested in finding the store, where unicycles are selling. I had a talk with that organizer that was mentioned before and visited such store. I even decided which unicycle I wanted at most, but I did not buy it. I have returned to St.Petersburg in summer for a reason that absolutely was not connecting with unicycling. Under influence of some obstacles I changed back ticket to Minks for 10 days earlier, but bought that unicycle that I selected during visiting in winter.
\medskip\par It was hard to learn how to ride in a correct way on one wheel, one moment I tried to start moving from the hill, because it seems to be easier when some speed is already is. Also I was storing my unicycles at flats of my friends. So the regular trainings were quite problematic. But now 3 years are passed, I can ride a unicycle freely, with a bag, with juggling 3 balls or for a long distance for the unicycle with such diameter of the wheel: 20 km on Minsk streets is the current record. Even succeeded to ride previous winter in VDNH park.
\ \\[0.2 cm]
\hline
\ \\[0.2 cm]
{\large\texttt{Question 2}}
\\
\textit{Tell me about your best or worst school teacher.
}
\ \\[0.3cm]
\medskip\par 
When I was in 5th form, my class was formed entirely from children who had previously studied in different schools. Everything was quite new. I remember very well an impression that a teacher of the Russian language and literature made on us. She seemed strict, but how cool she was giving us a material. We learned the full rules from a separate book of the Russian language, not from a textbook for the 5th form. We wrote dictations with words based on exceptions to rules that were learned before. She also had a very impressive appearance. She was tall, with a long black braid, always with a beautiful and bright brooch and she also had a low-pitched voice. She left school after that year, but I'm sure that whole our class remembers her.
\ \\[0.2 cm]
\hline
\ \\[0.2 cm]
{\large\texttt{Question 3}}
\ \\[0.1cm]
\textit{What bands were popular when you were young?
}
\ \\[0.3cm]
\medskip\par 
This is the most difficult question. The only thing I remember well is how loudly everyone at summer camp shouted It's my life Bon Jovi. Still often sings of Ляписы, Кино or Машина времени were sang. But I don't think this is specific of that time. Rather, this songs are usually determined by the atmosphere and the presence of the guitar.
\ \\[0.5 cm]
\hline
\end{document}